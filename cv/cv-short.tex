%%%%%%%%%%%%%%%%%%%%%%%%%%%%%%%%%%%%%%%%%
% Medium Length Graduate Curriculum Vitae
% LaTeX Template
% Version 1.1 (9/12/12)
%
% This template has been downloaded from:
% http://www.LaTeXTemplates.com
%
% Original author:
% Rensselaer Polytechnic Institute (http://www.rpi.edu/dept/arc/training/latex/resumes/)
%
% Important note:
% This template requires the res.cls file to be in the same directory as the
% .tex file. The res.cls file provides the resume style used for structuring the
% document.
%
%%%%%%%%%%%%%%%%%%%%%%%%%%%%%%%%%%%%%%%%%

%----------------------------------------------------------------------------------------
%	PACKAGES AND OTHER DOCUMENT CONFIGURATIONS
%----------------------------------------------------------------------------------------

\documentclass[margin, 10pt]{res-short} % Use the res.cls style, the font size can be changed to 11pt or 12pt here

\usepackage{helvet} % Default font is the helvetica postscript font
%\usepackage{newcent} % To change the default font to the new century schoolbook postscript font uncomment this line and comment the one above



\usepackage{hyperref}
\hypersetup{
    colorlinks=true,
    linkcolor=blue,
    filecolor=magenta,      
    urlcolor=cyan,
}
\urlstyle{same}

\usepackage{etaremune,amsmath,amssymb}

\setlength{\textwidth}{5.1in} % Text width of the document
%\setlength{\sectionskip}{.2in}

\newsectionwidth{1.4in}
\hoffset 1in

\begin{document}

%----------------------------------------------------------------------------------------
%	NAME AND ADDRESS SECTION
%----------------------------------------------------------------------------------------

\moveleft.7\hoffset\centerline{\huge\bf Maoyuan `Raymond' Song} % Your name at the top
 
\moveleft\hoffset\vbox{\hrule width 0.9\resumewidth height 1pt}\smallskip % Horizontal line after name; adjust line thickness by changing the '1pt'
 
\moveleft.7\hoffset\centerline{Department of Computer Science, Purdue University} % Your address
\moveleft.7\hoffset\centerline{305 N. University St, West Lafayette, IN 47907}

%----------------------------------------------------------------------------------------

\begin{resume}

\section{CONTACT}
\emph{Email}: MaoyuanRS@gmail.com\\
\emph{Personal Page}: \href{https://maoyuans.github.io}{maoyuans.github.io}

 
\section{RESEARCH \\ INTERESTS}  

Online algorithms; Learning-augmented algorithms; Sublinear-time and sublinear-space algorithms; Statistical estimation; Computational complexity; Beyond worst-case analysis; Learning theory.

I am interested in the intersection of machine learning, artificial intelligence, and classical algorithms: How to use classical algorithms to augment machine learning and artificial intelligence, and how to use machine learning methods to facilitate classical algorithms, to solve theoretical and practical challenges.

\section{SKILLS}

{\bf Programming Languages:} Python, C, C++, LaTeX, Java, Git.\\
{\bf Languages:} English (Fluent), Mandarin Chinese (Native).
 
\section{EDUCATION}

{\bf Purdue University} \hfill West Lafayette, IN \\
\emph{Ph.D. in Computer Science} \hfill August 2020 - Present\\
Advised by: Elena Grigorescu and Paul Valiant.\\
Relevant Coursework: Machine Learning Theory, Cryptography, Sublinear Algorithms, Randomized Algorithms, Theory of Computation.

{\bf Carnegie Mellon University} \hfill Pittsburgh, PA \\
\emph{M.S. in Computer Science} \hfill May 2019 - May 2020 \\
Advised by: Carleton Kingsford.\\
Thesis Title: Linear Time Addition of Fibonacci Encodings.

{\bf Carnegie Mellon University} \hfill Pittsburgh, PA \\
\emph{B.S. in Computer Science} \hfill August 2015 - May 2019 \\
Minor in Discrete Math \& Logic, graduated with University Honors. \\
Relevant Coursework: Algorithm Design \& Analysis, Machine Learning (PhD), Spectral Graph Theory, Set Theory, Extremal Combinatorics.




%\section{EMPLOYMENT}

%{\bf Senior Project Member, Content Manager} \hfill January 2018 - May 2020 \\
%Carnegie Mellon University Computer Science Academy \hfill Pittsburgh, PA

%\begin{itemize}
%\item Participated as a senior member in the development of CMU Computer Science Academy, a university-sponsored non-profit organization aiming to provide accessible and effective experiences with CS for highschool students and educators.
%\item Created and managed contents including student exercises, quality assurance, and support resources for educators.
%\end{itemize} 


\section{PUBLICATIONS}
{\it Authors are ordered alphabetically, as is common practice in theoretical computer science.}
\begin{etaremune}
\item Learning-Augmented Algorithms for Online Covering Programs with Convex Objectives.\\
Elena Grigorescu, Young-San Lin, {\bf Maoyuan Song}.\\
\emph{In submission}.
\item A Simple Learning-Augmented Algorithm for Online Packing with Concave Objectives.\\
Elena Grigorescu, Young-San Lin, {\bf Maoyuan Song}.\\
\emph{arXiv preprint arXiv:2406.03754, 2024}.
\item All-Purpose Mean Estimation over $\mathbb{R}$: Optimal Sub-Gaussianity with Outlier Robustness and Low Moments Performance.\\
Jasper C.H. Lee, Walter McKelvie, {\bf Maoyuan Song}, Paul Valiant.\\
\emph{In submission}.
\item Optimality in Mean Estimation: Beyond Worst-Case, Beyond Sub-Gaussian, Beyond $1 + \alpha$ Moments.\\
Trung Dang, Jasper C.H. Lee, {\bf Maoyuan Song}, Paul Valiant.\\
\emph{Conference on Neural Information Processing Systems (NeurIPS)} (2023).
\item Learning-Augmented Algorithms for Online Linear and Semidefinite Programming.\\
Elena Grigorescu, Young-San Lin, Sandeep Silwal, {\bf Maoyuan Song}, Samson Zhou.\\
\emph{Conference on Neural Information Processing Systems (NeurIPS)} (2022). Selected for spotlight presentation.
%\item Linear Time Addition of Fibonacci Encodings.\\
%{\bf Maoyuan (Raymond) Song}.\\
%\emph{Master's Thesis} (2020). 
%\item Application of Convolutional Neural Networks in Accent Identification.\\
%Kevin Chionh, {\bf Maoyuan Song}, Yue Yin.\\
%\emph{Project Report} (2018).
\end{etaremune}

\section{INVITED \\ PROGRAMS}

{\bf Simons Institute for the Theory of Computing, UC Berkeley} \hfill Berkeley, CA \\
\emph{Error-Correcting Codes: Theory and Practice} \hfill January 2024 - March 2024 

\section{INVITED \\ TALKS}

Simple Switching Strategies for Learning-Augmented Algorithms.
\begin{itemize}
    \item TTIC Workshop on Learning-Augmented Algorithms, August 2024.
\end{itemize}

Beyond Worst-Case Optimality in Mean Estimation.
\begin{itemize}
    \item Conference on Neural Information Processing Systems (NeurIPS), December 2023.
    \item Carnegie Mellon University Theory Lunch, September 2023.
    \item Rutgers/DIMACS Theory of Computing Seminar, September 2023.
    \item Northwestern Theory Seminar, July 2023.
\end{itemize}

Learning-Augmented Algorithms for Online Linear and Semidefinite Programming.
\begin{itemize}
\item Conference on Neural Information Processing Systems (NeurIPS), December 2022.
\end{itemize}
%\item Learning-Augmented Algorithms for Online General Covering LPs.\\
%Theory Reading Group at Purdue, November 2022.
%\item On `The Primal-Dual Method for Learning Augmented Algorithms'.\\
%Theory Reading Group at Purdue, February 2022.
%\item On `PROPm Allocations of Indivisible Goods to Multiple Agents'.\\
%Theory Reading Group at Purdue, November 2021.
%\item Online Facility Location Problem with Recourse.\\
%Theory Reading Group at Purdue, March 2021.
%\item Fields and Polynomials, based on 15-751 TCS Toolkit.\\
%Advanced Algorithm Reading Group at Purdue, October 2020.
%\item Fast Multiplication using Discrete Fourier Transform, based on 15-751 TCS Toolkit.\\
%Advanced Algorithm Reading Group at Purdue, September 2020.

\section{OTHER \\ ACTIVITIES}

{\bf Carnegie Mellon University, Kingsford Labs} \hfill May 2018 - August 2018\\
Developed and optimized \emph{salmon}, a genetic quantification and alignment software using machine learning. Introduced speed-ups via parallelization using NVIDIA's CUDA library in C++.

{\bf Carnegie Mellon University, Computer Science Academy} \hfill January 2018 - May 2020\\
Participated in the design and development of CMU Computer Science Academy, a university-sponsored online curriculum platform for K-12 computer science education. Visited six highschools in the Greater Pittsburgh area as practical and educational support specialist.

\section{AWARDS}
{\bf Purdue Research Fundation Ross-Lynn Research Scholars Grant}. \hfill Fall~2022


\end{resume}
\end{document}