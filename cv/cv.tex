%%%%%%%%%%%%%%%%%%%%%%%%%%%%%%%%%%%%%%%%%
% Medium Length Graduate Curriculum Vitae
% LaTeX Template
% Version 1.1 (9/12/12)
%
% This template has been downloaded from:
% http://www.LaTeXTemplates.com
%
% Original author:
% Rensselaer Polytechnic Institute (http://www.rpi.edu/dept/arc/training/latex/resumes/)
%
% Important note:
% This template requires the res.cls file to be in the same directory as the
% .tex file. The res.cls file provides the resume style used for structuring the
% document.
%
%%%%%%%%%%%%%%%%%%%%%%%%%%%%%%%%%%%%%%%%%

%----------------------------------------------------------------------------------------
%	PACKAGES AND OTHER DOCUMENT CONFIGURATIONS
%----------------------------------------------------------------------------------------

\documentclass[margin, 10pt]{res} % Use the res.cls style, the font size can be changed to 11pt or 12pt here

\usepackage{helvet} % Default font is the helvetica postscript font
%\usepackage{newcent} % To change the default font to the new century schoolbook postscript font uncomment this line and comment the one above



\usepackage{hyperref}
\hypersetup{
    colorlinks=true,
    linkcolor=blue,
    filecolor=magenta,      
    urlcolor=cyan,
}
\urlstyle{same}

\usepackage{etaremune}

\setlength{\textwidth}{5.1in} % Text width of the document


\begin{document}

%----------------------------------------------------------------------------------------
%	NAME AND ADDRESS SECTION
%----------------------------------------------------------------------------------------

\moveleft.5\hoffset\centerline{\huge\bf Maoyuan `Raymond' Song} % Your name at the top
 
\moveleft\hoffset\vbox{\hrule width\resumewidth height 1pt}\smallskip % Horizontal line after name; adjust line thickness by changing the '1pt'
 
\moveleft.5\hoffset\centerline{Department of Computer Science, Purdue University} % Your address
\moveleft.5\hoffset\centerline{305 N. University St, West Lafayette, IN 47907}

%----------------------------------------------------------------------------------------

\begin{resume}

\section{CONTACT}
\emph{Email}: MaoyuanRS (at) gmail (dot) com\\
\emph{Personal Page}: \href{https://maoyuans.github.io}{maoyuans.github.io}

 
\section{RESEARCH \\ INTERESTS}  

Online algorithms; Beyond worst-case analysis; Learning-augmented algorithms; Sublinear-time and sublinear-space algorithms; Computational complexity; Learning theory.

 
\section{EDUCATION}

{\bf Ph.D. Student in Computer Science} \hfill August 2020 - Present \\
Purdue University \hfill West Lafayette, IN

\begin{itemize}
\item Advisors: Elena Grigorescu, Paul Valiant.
\end{itemize} 

{\bf M.S. in Computer Science} \hfill May 2019 - May 2020 \\
Carnegie Mellon University \hfill Pittsburgh, PA

\begin{itemize}
\item Advisor: Carleton Kingsford. \hspace{-2em}
\item Thesis: Linear Time Addition of Fibonacci Encodings.
\end{itemize} 

{\bf B.S. in Computer Science} \hfill Aug 2015 - May 2020 \\
Carnegie Mellon University \hfill Pittsburgh, PA

\begin{itemize}
\item Minor: Discrete Mathematics and Logic.
\item Graduated with University Honors.
\end{itemize} 


\section{EMPLOYMENT}

{\bf Senior Project Member, Content Manager} \hfill January 2018 - May 2020 \\
Carnegie Mellon University Computer Science Academy \hfill Pittsburgh, PA

\begin{itemize}
\item Participated as a senior member in the development of CMU Computer Science Academy, a university-sponsored non-profit organization aiming to provide accessible and effective experiences with CS for highschool students and educators.
\item Created and managed contents including student exercises, quality assurance, and support resources for educators.
\end{itemize} 


\section{PUBLICATIONS}
\begin{etaremune}
\item Optimality in Mean Estimation: Beyond Worst-Case, Beyond Sub-Gaussian, Beyond $1 + \alpha$ Moments.\\
Trung Dang, Jasper C.H. Lee, {\bf Maoyuan Song}, Paul Valiant.\\
\emph{Conference on Neural Information Processing Systems (NeurIPS)} (2023).
\item Learning-Augmented Algorithms for Online Linear and Semidefinite Programming.\\
Elena Grigorescu, Young-San Lin, Sandeep Silwal, {\bf Maoyuan Song}, Samson Zhou.\\
\emph{Conference on Neural Information Processing Systems (NeurIPS)} (2022). Selected for spotlight presentation.
\item Linear Time Addition of Fibonacci Encodings.\\
{\bf Maoyuan (Raymond) Song}.\\
\emph{Master's Thesis} (2020). 
%\item Application of Convolutional Neural Networks in Accent Identification.\\
%Kevin Chionh, {\bf Maoyuan Song}, Yue Yin.\\
%\emph{Project Report} (2018).
\end{etaremune}


\section{TEACHING}

{\bf Purdue University, Department of Computer Science}\\
Graduate Teaching Assistant

\begin{itemize}
\item CS588 Randomized Algorithms \hfill Spring 2022
\item CS584 Theory of Computation \hfill Fall 2021
\item CS381 Introduction to the Analysis of Algorithms \hfill Spring 2021
\item CS251 Data Structures and Algorithms \hfill Fall 2020
\end{itemize}
{\bf Carnegie Mellon University, Department of Computer Science}\\
Graduate Teaching Assistant

\begin{itemize}
\item 15-451 Algorithm Design and Analysis \hfill Spring 2020, Fall 2019
\end{itemize}

{\bf Carnegie Mellon University}\\
Student-Led Course Instructor

\begin{itemize}
\item 98-205 StuCo: Introduction to Minecraft \hfill Fall 2016 - Spring 2019
\end{itemize}

\section{AWARDS}
{\bf Purdue Research Fundation Ross-Lynn Research Scholars Grant}. \hfill Fall~2022

\section{PROFESSIONAL \\ SERVICE}

{\bf External Conference Reviewer}

\begin{itemize}
\item ACM Symposium on Theory of Computing (STOC) 2023.
\item Innovations in Theoretical Computer Science (ITCS) 2023, 2022.
\item Conference on Neural Information Processing Systems (NeurIPS) 2022, 2021.
\item International Symposium on Symbolic and Numeric Algorithms for Scientific Computing (SYNASC) 2023, 2022.
\item Journal of Artificial Intelligence Research (JAIR) 2022.
\end{itemize}
{\bf Organizer}

\begin{itemize}
\item TCS Reading Group at Purdue, Fall 2023
\item Theoretical Computer Science Seminar at Purdue, Fall 2022 - Fall 2023.
\item Advanced Algorithm Reading Group at Purdue, Fall 2020.
\end{itemize}

\section{TALKS and \\ PRESENTATIONS}
\begin{itemize}
\item Beyond Worst-Case Optimality in Mean Estimation.\\
Carnegie Mellon University Theory Lunch, Sept 2023.
\item Beyond Worst-Case Optimality in Mean Estimation.\\
Rutgers/DIMACS Theory of Computing Seminar, Sept 2023.
\item Beyond Worst-Case Optimality in Mean Estimation.\\
Northwestern Theory Seminar, July 2023.
\item Learning-Augmented Algorithms for Online Linear and Semidefinite Program-
ming.\\
Conference on Neural Information Processing Systems (NeurIPS), December 2022.
\item Learning-Augmented Algorithms for Online General Covering LPs.\\
Theory Reading Group at Purdue, November 2022.
%\item On `The Primal-Dual Method for Learning Augmented Algorithms'.\\
%Theory Reading Group at Purdue, February 2022.
%\item On `PROPm Allocations of Indivisible Goods to Multiple Agents'.\\
%Theory Reading Group at Purdue, November 2021.
\item Online Facility Location Problem with Recourse.\\
Theory Reading Group at Purdue, March 2021.
%\item Fields and Polynomials, based on 15-751 TCS Toolkit.\\
%Advanced Algorithm Reading Group at Purdue, October 2020.
%\item Fast Multiplication using Discrete Fourier Transform, based on 15-751 TCS Toolkit.\\
%Advanced Algorithm Reading Group at Purdue, September 2020.
\item Linear Time Addition of Fibonacci Encodings.\\
Master's Thesis Defense, April 2020.
\end{itemize}


\end{resume}
\end{document}