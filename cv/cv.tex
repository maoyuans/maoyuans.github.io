%%%%%%%%%%%%%%%%%%%%%%%%%%%%%%%%%%%%%%%%%
% Medium Length Graduate Curriculum Vitae
% LaTeX Template
% Version 1.1 (9/12/12)
%
% This template has been downloaded from:
% http://www.LaTeXTemplates.com
%
% Original author:
% Rensselaer Polytechnic Institute (http://www.rpi.edu/dept/arc/training/latex/resumes/)
%
% Important note:
% This template requires the res.cls file to be in the same directory as the
% .tex file. The res.cls file provides the resume style used for structuring the
% document.
%
%%%%%%%%%%%%%%%%%%%%%%%%%%%%%%%%%%%%%%%%%

%----------------------------------------------------------------------------------------
%	PACKAGES AND OTHER DOCUMENT CONFIGURATIONS
%----------------------------------------------------------------------------------------

\documentclass[margin, 10pt]{res} % Use the res.cls style, the font size can be changed to 11pt or 12pt here

\usepackage{helvet} % Default font is the helvetica postscript font
%\usepackage{newcent} % To change the default font to the new century schoolbook postscript font uncomment this line and comment the one above



\usepackage{hyperref}
\hypersetup{
    colorlinks=true,
    linkcolor=blue,
    filecolor=magenta,      
    urlcolor=cyan,
}
\urlstyle{same}

\usepackage{etaremune,amsmath,amssymb}

\setlength{\textwidth}{5.1in} % Text width of the document


\begin{document}

%----------------------------------------------------------------------------------------
%	NAME AND ADDRESS SECTION
%----------------------------------------------------------------------------------------

\moveleft.5\hoffset\centerline{\huge\bf Maoyuan `Raymond' Song} % Your name at the top
 
\moveleft\hoffset\vbox{\hrule width\resumewidth height 1pt}\smallskip % Horizontal line after name; adjust line thickness by changing the '1pt'
 
\moveleft.5\hoffset\centerline{Department of Computer Science, Purdue University} % Your address
\moveleft.5\hoffset\centerline{305 N. University St, West Lafayette, IN 47907}

%----------------------------------------------------------------------------------------

\begin{resume}

\section{CONTACT}
\emph{Email}: MaoyuanRS (at) gmail (dot) com\\
\emph{Personal Page}: \href{https://maoyuans.github.io}{maoyuans.github.io}

 
\section{RESEARCH \\ INTERESTS}  

Online algorithms; Learning-augmented algorithms; Statistical estimation; Data-dependent algorithm design and analysis; Sublinear-time and sublinear-space algorithms; Beyond worst-case analysis; Computational complexity; Learning theory.

Broadly, the intersection of machine learning, artificial intelligence, and classical algorithms: How to combine the robustness of classical algorithms and the performance of machine learning, to achieve fairness, efficiency, and reliability.
 
\section{EDUCATION}

{\bf Purdue University} \hfill West Lafayette, IN\\
{\it Ph.D. Candidate in Computer Science} \hfill August 2020 - Present\\
\hphantom{12}- Advisors: Elena Grigorescu, Paul Valiant.\\
\hphantom{12}- Preliminary examination passed, planning to graduate on May 2025.
%\begin{itemize}
%\item Advisors: Elena Grigorescu, Paul Valiant.
%\item Preliminary examination passed, planning to graduate on May 2025.
%\end{itemize} 

{\bf Carnegie Mellon University} \hfill Pittsburgh, PA\\
{\it M.S. in Computer Science} \hfill May 2019 - May 2020 \\
\hphantom{12}- Advisor: Carleton Kingsford.\\
\hphantom{12}- Thesis: Linear Time Addition of Fibonacci Encodings.
%\begin{itemize}
%\item Advisor: Carleton Kingsford. \hspace{-2em}
%\item Thesis: Linear Time Addition of Fibonacci Encodings.
%\end{itemize} 

{\bf Carnegie Mellon University} \hfill Pittsburgh, PA\\
{\it B.S. in Computer Science} \hfill Aug 2015 - May 2019 \\
\hphantom{12}- Minor: Discrete Mathematics and Logic.\\
\hphantom{12}- Graduated with University Honors.
%\begin{itemize}
%\item Minor: Discrete Mathematics and Logic.
%\item Graduated with University Honors.
%\end{itemize} 


%\section{EMPLOYMENT}

%{\bf Senior Project Member, Content Manager} \hfill January 2018 - May 2020 \\
%Carnegie Mellon University Computer Science Academy \hfill Pittsburgh, PA

%\begin{itemize}
%\item Participated as a senior member in the development of CMU Computer Science Academy, a university-sponsored non-profit organization aiming to provide accessible and effective experiences with CS for highschool students and educators.
%\item Created and managed contents including student exercises, quality assurance, and support resources for educators.
%\end{itemize} 


\section{PUBLICATIONS}
{\it Authors are ordered alphabetically, as is common practice in theoretical computer science.}
\begin{etaremune}
\item Learning-Augmented Algorithms for Online Concave Packing and Convex Covering.\\
Elena Grigorescu, Young-San Lin, {\bf Maoyuan Song}.\\
\emph{arXiv preprint arXiv:2411.08332, 2024}.
\item A Simple Learning-Augmented Algorithm for Online Packing with Concave Objectives.\\
Elena Grigorescu, Young-San Lin, {\bf Maoyuan Song}.\\
\emph{arXiv preprint arXiv:2406.03754, 2024}.
\item All-Purpose Mean Estimation over $\mathbb{R}$: Optimal Sub-Gaussianity with Outlier Robustness and Low Moments Performance.\\
Jasper C.H. Lee, Walter McKelvie, {\bf Maoyuan Song}, Paul Valiant.\\
\emph{In submission}.
\item Optimality in Mean Estimation: Beyond Worst-Case, Beyond Sub-Gaussian, Beyond $1 + \alpha$ Moments.\\
Trung Dang, Jasper C.H. Lee, {\bf Maoyuan Song}, Paul Valiant.\\
\emph{Conference on Neural Information Processing Systems (NeurIPS)} (2023).
\item Learning-Augmented Algorithms for Online Linear and Semidefinite Programming.\\
Elena Grigorescu, Young-San Lin, Sandeep Silwal, {\bf Maoyuan Song}, Samson Zhou.\\
\emph{Conference on Neural Information Processing Systems (NeurIPS)} (2022). Selected for spotlight presentation.
\item Linear Time Addition of Fibonacci Encodings.\\
{\bf Maoyuan (Raymond) Song}.\\
\emph{Master's Thesis} (2020). 
%\item Application of Convolutional Neural Networks in Accent Identification.\\
%Kevin Chionh, {\bf Maoyuan Song}, Yue Yin.\\
%\emph{Project Report} (2018).
\end{etaremune}

\section{INVITED \\ PROGRAMS}

{\bf Simons Institute for the Theory of Computing, UC Berkeley} \hfill Berkeley, CA \\
\emph{Error-Correcting Codes: Theory and Practice} \hfill January 2024 - March 2024 

\section{INVITED \\ TALKS}

Learning-Augmented Algorithms for Online Concave Packing and Convex Covering.\\
\hphantom{12}- Purdue Theory Seminar, October 2024.\\
\hphantom{12}- UIUC Theory Seminar, October 2024.

Simple Switching Strategies for Learning-Augmented Algorithms.\\
\hphantom{12}- TTIC Workshop on Learning-Augmented Algorithms, August 2024.

Beyond Worst-Case Optimality in Mean Estimation.\\
\hphantom{12}- Conference on Neural Information Processing Systems (NeurIPS), December 2023.\\
\hphantom{12}- Carnegie Mellon University Theory Lunch, September 2023.\\
\hphantom{12}- Rutgers/DIMACS Theory of Computing Seminar, September 2023.\\
\hphantom{12}- Northwestern Theory Seminar, July 2023.

Learning-Augmented Algorithms for Online Linear and Semidefinite Programming.\\
\hphantom{12}- Conference on Neural Information Processing Systems (NeurIPS), December 2022.
%\item Learning-Augmented Algorithms for Online General Covering LPs.\\
%Theory Reading Group at Purdue, November 2022.
%\item On `The Primal-Dual Method for Learning Augmented Algorithms'.\\
%Theory Reading Group at Purdue, February 2022.
%\item On `PROPm Allocations of Indivisible Goods to Multiple Agents'.\\
%Theory Reading Group at Purdue, November 2021.
%\item Online Facility Location Problem with Recourse.\\
%Theory Reading Group at Purdue, March 2021.
%\item Fields and Polynomials, based on 15-751 TCS Toolkit.\\
%Advanced Algorithm Reading Group at Purdue, October 2020.
%\item Fast Multiplication using Discrete Fourier Transform, based on 15-751 TCS Toolkit.\\
%Advanced Algorithm Reading Group at Purdue, September 2020.


\section{PROFESSIONAL \\ SERVICE}

{\bf External Conference Reviewer}\\
\hphantom{12}- International Conference on Artificial Intelligence and Statistics (AISTATS) 2025.\\
\hphantom{12}- SIAM Symposium on Simplicity in Algorithms (SOSA) 2025.\\
\hphantom{12}- The European Symposium on Algorithms (ESA) 2024.\\
\hphantom{12}- International Symposium on Theoretical Aspects of Computer Science (STACS) \hphantom{1234} 2024.\\
\hphantom{12}- ACM Symposium on Theory of Computing (STOC) 2025, 2024, 2023.\\
\hphantom{12}- Conference on Neural Information Processing Systems (NeurIPS) 2024, 2022, \hphantom{1234} 2021.\\
\hphantom{12}- Innovations in Theoretical Computer Science (ITCS) 2023, 2022.\\
\hphantom{12}- International Symposium on Symbolic and Numeric Algorithms for Scientific\\ \hphantom{1234} Computing (SYNASC) 2023, 2022.\\
\hphantom{12}- Journal of Artificial Intelligence Research (JAIR) 2022.

{\bf Organizer}\\
\hphantom{12}- TCS Reading Group at Purdue, Spring 2025, Fall 2023.\\
\hphantom{12}- Theoretical Computer Science Seminar at Purdue, Fall 2023 - Fall 2022.\\
\hphantom{12}- Advanced Algorithm Reading Group at Purdue, Fall 2020.


\section{PROFESSIONAL \\ ACTIVITIES}

{\bf Purdue University, Department of Computer Science}\\
Graduate Teaching Assistant\\
\hphantom{12}- CS588 Randomized Algorithms \hfill Spring 2022\\
\hphantom{12}- CS584 Theory of Computation \hfill Fall 2021\\
\hphantom{12}- CS381 Introduction to the Analysis of Algorithms \hfill Fall 2024, Spring 2021\\
\hphantom{12}- CS251 Data Structures and Algorithms \hfill Fall 2020


{\bf Carnegie Mellon University, Department of Computer Science}\\
Graduate Teaching Assistant\\
\hphantom{12}- 15-451 Algorithm Design and Analysis \hfill Spring 2020, Fall 2019


{\bf Kingsford Group, Carnegie Mellon University}\\
Student researcher \hfill Summer 2018

{\bf Carnegie Mellon University Computer Science Academy}\\
Senior Project Member, Content Manager \hfill Spring 2018 - Spring 2020

{\bf Carnegie Mellon University}\\
Student-Led Course Instructor\\
\hphantom{12}- 98-205 StuCo: Introduction to Minecraft \hfill Fall 2016 - Spring 2019

\section{AWARDS}
{\bf Purdue Research Fundation Ross-Lynn Research Scholars Grant}. \hfill Fall~2022


\end{resume}
\end{document}