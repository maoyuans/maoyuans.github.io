%%%%%%%%%%%%%%%%%%%%%%%%%%%%%%%%%%%%%%%%%
% Medium Length Graduate Curriculum Vitae
% LaTeX Template
% Version 1.1 (9/12/12)
%
% This template has been downloaded from:
% http://www.LaTeXTemplates.com
%
% Original author:
% Rensselaer Polytechnic Institute (http://www.rpi.edu/dept/arc/training/latex/resumes/)
%
% Important note:
% This template requires the res.cls file to be in the same directory as the
% .tex file. The res.cls file provides the resume style used for structuring the
% document.
%
%%%%%%%%%%%%%%%%%%%%%%%%%%%%%%%%%%%%%%%%%

%----------------------------------------------------------------------------------------
%	PACKAGES AND OTHER DOCUMENT CONFIGURATIONS
%----------------------------------------------------------------------------------------

\documentclass[margin, 10pt]{res} % Use the res.cls style, the font size can be changed to 11pt or 12pt here

\usepackage{helvet} % Default font is the helvetica postscript font
%\usepackage{newcent} % To change the default font to the new century schoolbook postscript font uncomment this line and comment the one above
\usepackage[UTF8]{ctex}


\usepackage{hyperref}
\hypersetup{
    colorlinks=true,
    linkcolor=blue,
    filecolor=magenta,      
    urlcolor=cyan,
}
\urlstyle{same}

\usepackage{etaremune,amsmath,amssymb}

\setlength{\textwidth}{5.1in} % Text width of the document


\begin{document}

%----------------------------------------------------------------------------------------
%	NAME AND ADDRESS SECTION
%----------------------------------------------------------------------------------------

\moveleft.5\hoffset\centerline{\huge\bf 宋 茂源 \;\textbar\; Maoyuan `Raymond' Song} % Your name at the top
 
\moveleft\hoffset\vbox{\hrule width\resumewidth height 1pt}\smallskip % Horizontal line after name; adjust line thickness by changing the '1pt'
 
\moveleft.5\hoffset\centerline{普渡大学 计算机科学系} % Your address
\moveleft.5\hoffset\centerline{305 N. University St, West Lafayette, IN 47907}

%----------------------------------------------------------------------------------------

\begin{resume}

\section{联系方式}
\emph{电子邮箱}: MaoyuanRS @ gmail . com\\
\emph{个人主页}: \href{https://maoyuans.github.io}{maoyuans.github.io}

 
\section{研究方向}  

在线算法 (Online algorithm); 学习增强算法 (Learning-augmented algorithm); 次线性算法 (Sublinear algorithm); 统计估计 (Statistical estimation); 计算复杂性理论 (Computational complexity); 超越最坏情况分析 (Beyond worst-case analysis); 学习理论 (Learning theory).

广泛地来说, 机器学习, 人工智能, 与传统算法的交叉领域: 如何用传统算法辅助机器学习与人工智能, 及如何用机器学习方法辅助传统算法, 确保公平, 效率, 和可靠性.

\section{教育背景}

{\bf 计算机科学\; 博士候选} \hfill 2020年8月 -- \\
普渡大学 \hfill 美国印第安纳州西拉法叶

\begin{itemize}
\item 导师: Elena Grigorescu,Paul Valiant.
\item 已通过资格考试. 预计2025年5月毕业.
\end{itemize} 

{\bf 受邀研究员} \hfill 2024年1月 -- 2024年3月 \\
Simons计算理论学院, 加州大学伯克利分校 \hfill 美国加利福尼亚州伯克利
\begin{itemize}
\item 参与项目: Error-Correcting Codes: Theory and Practice.
\end{itemize}

{\bf 计算机科学\; 硕士学位} \hfill 2019年5月 -- 2020年5月 \\
卡内基梅隆大学 \hfill 美国宾夕法尼亚州匹兹堡

\begin{itemize}
\item 导师: Carleton Kingsford. \hspace{-2em}
\item 论文题目: Linear Time Addition of Fibonacci Encodings.\\
研究如何用线性时间在不解码的情况下将斐波那契编码求和。
\end{itemize} 

{\bf 计算机科学\; 学士学位} \hfill 2015年8月 -- 2019年5月 \\
卡内基梅隆大学 \hfill 美国宾夕法尼亚州匹兹堡

\begin{itemize}
\item 辅修专业: 离散数学与逻辑.
\item 校级优秀毕业生 (University Honors).
\end{itemize} 

{\bf 高中学校} \hfill 2010年9月 -- 2014年6月 \\
北京市第八中学超常儿童教育实验班 (北京八中少儿班) 17班 \hfill 中国北京市西城区


%\section{实习经历}

%{\bf 资深成员,项目内容负责人} \hfill 2018年1月 - 2020年5月 \\
%Carnegie Mellon University Computer Science Academy \hfill 美国宾夕法尼亚州匹兹堡

%\begin{itemize}
%\item 作为资深项目成员参与卡内基梅隆大学计算机学院的CMU Computer Science Academy项目。Computer Science Academy是一个由计算机学院官方资助的非盈利性组织,致力于为美国高中的学生与教师提供高效且便捷的计算机科学教育资源。
%\item 设计并管理项目内容,包括但不限于课程练习,质量保证,和教育工作者支持资源。
%\end{itemize} 


\section{论文发表}
{\it 遵循理论计算机科学领域惯例, 文章作者按照姓氏首字母排序.}
\begin{etaremune}
\item 应用于优化目标为凸函数的在线覆盖规划的学习增强算法.\\
Learning-Augmented Algorithms for Online Covering Programs with Convex Objectives.\\
Elena Grigorescu, Young-San Lin, {\bf Maoyuan Song}.\\
\emph{文章审阅中.}
\item 一个应用于优化目标为凹函数的在线装箱规划的简洁学习增强算法.\\
A Simple Learning-Augmented Algorithm for Online Packing with Concave Objectives.\\
Elena Grigorescu, Young-San Lin, {\bf Maoyuan Song}.\\
\emph{文章审阅中.}
\item 一维实数轴上的通用均值估算: 最优次高斯表现, 耐错性, 与重尾分布表现.\\
All-Purpose Mean Estimation over $\mathbb{R}$: Optimal Sub-Gaussianity with Outlier Robustness and Low Moments Performance.\\
Jasper C.H. Lee, Walter McKelvie, {\bf Maoyuan Song}, Paul Valiant.\\
\emph{文章审阅中.}
\item 均值估算的最优性: 超越最坏情况分析, 超越次高斯表现, 超越$1 + \alpha$阶动差.\\
Optimality in Mean Estimation: Beyond Worst-Case, Beyond Sub-Gaussian, Beyond $1 + \alpha$ Moments.\\
Trung Dang, Jasper C.H. Lee, {\bf Maoyuan Song}, Paul Valiant.\\
\emph{发表于 Conference on Neural Information Processing Systems (NeurIPS)} (2023).
\item 应用于在线线性与半正定规划的学习增强算法.\\
Learning-Augmented Algorithms for Online Linear and Semidefinite Programming.\\
Elena Grigorescu, Young-San Lin, Sandeep Silwal, {\bf Maoyuan Song}, Samson Zhou.\\
\emph{发表于 Conference on Neural Information Processing Systems (NeurIPS)} (2022), \emph{被选为重点展示 (Spotlight presentation)}.
\item 斐波那契编码的线性时间求和.\\
Linear Time Addition of Fibonacci Encodings.\\
{\bf Maoyuan (Raymond) Song}.\\
\emph{硕士论文} (2020).
%\item Application of Convolutional Neural Networks in Accent Identification.\\
%Kevin Chionh, {\bf Maoyuan Song}, Yue Yin.\\
%\emph{Project Report} (2018).
\end{etaremune}

\section{受邀讲座}
\begin{itemize}
\item Beyond Worst-Case Optimality in Mean Estimation.
\begin{itemize}
    \item Conference on Neural Information Processing Systems (NeurIPS), 2023年12月.
    \item Carnegie Mellon University Theory Lunch, 2023年9月.
    \item Rutgers/DIMACS Theory of Computing Seminar, 2023年9月.
    \item Northwestern Theory Seminar, 2023年7月.
\end{itemize}

\item Learning-Augmented Algorithms for Online Linear and Semidefinite Program-
ming.\\
Conference on Neural Information Processing Systems (NeurIPS), 2022年12月.
%\item Learning-Augmented Algorithms for Online General Covering LPs.\\
%Theory Reading Group at Purdue, 2022年11月.
%\item On `The Primal-Dual Method for Learning Augmented Algorithms'.\\
%Theory Reading Group at Purdue, February 2022.
%\item On `PROPm Allocations of Indivisible Goods to Multiple Agents'.\\
%Theory Reading Group at Purdue, November 2021.
%\item Online Facility Location Problem with Recourse.\\
%Theory Reading Group at Purdue, 2021年3月.
%\item Fields and Polynomials, based on 15-751 TCS Toolkit.\\
%Advanced Algorithm Reading Group at Purdue, October 2020.
%\item Fast Multiplication using Discrete Fourier Transform, based on 15-751 TCS Toolkit.\\
%Advanced Algorithm Reading Group at Purdue, September 2020.
\item Linear Time Addition of Fibonacci Encodings.\\
硕士论文答辩, 2020年4月.
\end{itemize}

\section{专业课程}
{\bf 普渡大学}
\begin{itemize}
    \item CS593 机器学习理论 (Machine Learning Theory)
    \item CS585 理论计算机科学工具 (TCS Toolkit)
    \item CS555 密码学 (Cryptography)
    \item CS590 次线性算法 (Sublinear Algorithms)
    \item CS590 随机算法 (Randomized Algorithms)
    \item CS584 计算复杂度理论 (Theory of Computation)
\end{itemize}

{\bf 卡内基梅隆大学}
\begin{itemize}
    \item 15859 谱图论 (Spectral Graph Theory)
    \item 21329 集合论 (Set Theory)
    \item 10701 机器学习 (Machine Learning)
    \item 21738 极值组合学 (Extremal Combinatorics)
    \item 15451 算法设计与分析 (Algorithm Design \& Analysis)
\end{itemize}

\section{学术活动}

{\bf 会议审稿人}

\begin{itemize}
\item The European Symposium on Algorithms (ESA) 2024
\item International Symposium on Theoretical Aspects of Comptuer Science (STACS) 2024
\item ACM Symposium on Theory of Computing (STOC) 2024, 2023
\item Conference on Neural Information Processing Systems (NeurIPS) 2024, 2022, 2021
\item Innovations in Theoretical Computer Science (ITCS) 2023, 2022
\item International Symposium on Symbolic and Numeric Algorithms for Scientific Computing (SYNASC) 2023, 2022
\item Journal of Artificial Intelligence Research (JAIR) 2022
\end{itemize}
{\bf 主办者}

\begin{itemize}
\item TCS Reading Group at Purdue, 2024年秋季, 2023年秋季.
\item Theoretical Computer Science Seminar at Purdue, 2023年秋季 -- 2022年秋季.
\item Advanced Algorithm Reading Group at Purdue, 2020年秋季.
\end{itemize}

\section{教学经验}

{\bf 普渡大学 计算机科学系}\\
研究生教学助理

\begin{itemize}
\item CS588 随机算法 (Randomized Algorithms) \hfill 2022年春季
\item CS584 计算复杂度理论 (Theory of Computation) \hfill 2021年秋季
\item CS381 算法分析入门 (Intro to the Analysis of Algorithms) \hfill 2021年春季
\item CS251 数据结构与算法 (Data Structures and Algorithms) \hfill 2020年秋季
\end{itemize}
{\bf 卡内基梅隆大学 计算机科学系}\\
研究生教学助理

\begin{itemize}
\item 15-451 算法设计与分析 (Algorithm Design \& Analysis) \\ \phantom{1}\hfill 2019年秋季 - 2020年春季
\end{itemize}

{\bf 卡内基梅隆大学 Computer Science Academy}\\
资深成员,项目内容负责人 \hfill 2018年春季 - 2020年春季

{\bf 卡内基梅隆大学}\\
学生开设课程教授

\begin{itemize}
\item 98-205 StuCo: Introduction to Minecraft \hfill 2016年秋季 -- 2019年春季
\end{itemize}

\section{所获奖项}
{\bf Purdue Research Fundation Ross-Lynn Research Scholars Grant} \\\hfill 2022年秋季 - 2023年春季


\end{resume}
\end{document}